\documentclass[12pt, a4paper]{article}
\usepackage{amsthm,amsfonts,amsmath,amssymb,amscd}
\usepackage[T2A]{fontenc}                         
\usepackage[utf8]{inputenc}                      
\usepackage[english, russian]{babel}
\usepackage{graphicx}
\usepackage{indentfirst}
\usepackage{multicol}
%\usepackage{cite} 
%\usepackage{psfrag}
%\IfFileExists{pscyr.sty}{\usepackage{pscyr}}{}    %

\usepackage[top=2cm,bottom=2cm,left=3cm,right=1.5cm]{geometry}

\linespread{1.5} 
\begin{document}
\begin{center}
Отчет о проделанной работе от 30.08.2022

\textbf{<<Разработка комбинированной математической модели распространения COVID-19 с учетом экономических агентов>>}

Воробьева Ирина
\end{center}
\section{Комбинированная модель распространения COVID-19 с учетом экономических агентов}

В рамках разрабатываемой модели предполагается объединить две модели: агентную модель распространения COVID-19 и модель распространения шоков в экономике. Первая модель позволяет моделировать распространение вируса <<снизу-вверх>>, учитывая взаимодействие агентов, их принадлежность к различным социальным группам и внешнее взаимодействие в виде локдаунов или дистанционного обучения. Вторая модель позволяет изучить влияние шоков, воздействующих на отдельные отрасли или комплексы отраслей, на экономическую систему в целом.

\subsection{Модель распространения шоков в экономике}
Данная модель уходит корнями к теории межотраслевых балансов Леонтьева \cite{Leontev}. В рамках данной модели предполагалось постоянство норм затрат на выпуск продукции в процессе межотраслевого взаимодействия. Со временем, теория Леонтьева потеряла свою актуальность в связи с возросшей взаимозаменяемостью товаров и услуг. Однако идея изучения взаимосвязей между отраслями не потеряла своей популярности, а гипотеза постоянства норм затрат преобразовалась в предположение о постоянстве структуры финансовых затрат в процессе производства товаров и услуг с учетом их отраслевой дифференциации. При использовании данного подхода рассматриваются производственные функции Кобба--Дугласа. В работах \cite{Inverse_Shan}, \cite{Duality_Shan} рассматриваются вопросы, посвященные поиску экономического равновесия в системах, описываемых матрицами межотраслевых затрат. В работе \cite{Akimova} проводится выявление отраслей--драйверов для экономики Российской Федерации и их влияние на ВВП страны.

\subsubsection{Таблица выпусков и затрат}

Рассмотрим матрицу $Z = \{Z^j_i\}_{i=1,\ldots,m+k}^{j=1,\ldots,m+n}$ --- матрицу межотраслевого взаимодействия, где $Z^j_i$ при $i = 1,\ldots,m,\ j=1,\ldots,m$ --- денежная сумма, полученная $i$-й отраслью от $j$-й отрасли за выполнение работы, $m$ --- количество отраслей, $n$ --- количество первичных ресурсов, $k$ --- количество конечных потребителей.

Обратной задачей в данном случае является построение модели распределения ресурсов в виде задачи выпуклого программирования, решение которой воспроизводит такую матрицу.

\subsubsection{Модель оптимального распределения ресурсов}
Рассмотрим группу из $m$ чистых отраслей, связанных через производственные факторы (ПФ). Обозначим $X_i^j$ объем  продукции $i$-й отрасли, необходимый для производства в $j$-й отрасли. Тогда $X^j = (X^j_1,\ldots X_m^j)$ --- затраты $j$-й отрасли на производство. Обозначим через $l^j = (l_1^j, \ldots , j_n^j)$ --- вектор первичных ресурсов, используемых $j$-й отраслью, всего первичных ресурсов $n$ видов.

Производственные функции отраслей $F_j(X^j, l^j)$ зависят от используемых в отрасли производственных факторов и первичных ресурсов. Будем считать, что производственные функции удовлетворяют неоклассическим предположениям: являются вогнутыми, монотонно неубывающими, непрерывными на $\mathbb{R}^{m+n}_+$ и равными нулю в нуле.

Внешние потребители получают объемы поставок $X^0 = (X_1^0,\ldots,X_m^0)$ от всех отраслей и имеют функцию полезности $F_0(X^0)$ также удовлетворяющую неоклассическим предположениям.

Тогда, можно поставить следующую задачу выпуклой оптимизации:

\begin{equation}\label{syst1}
\begin{aligned}
&F_0(X^0) \rightarrow \max;\\
&F_j(X^j, l^j) \geq \sum\limits_{i=0}^{m}X_j^i, j = 1,\ldots,m;\\
&\sum\limits_{j=1}^m l^j \leq l;\\
&X^0 \geq 0, \ldots, X^m \geq 0, l^1 \geq 0,\ldots,l^n \geq 0. 
\end{aligned}
\end{equation}

Для того, чтобы набор векторов $\{\hat{X}^0, \hat{X}^1, \ldots, \hat{X}^m, \hat{l}^1, \ldots, \hat{l^n}\}$, удовлетворяющий ограничениям \eqref{syst1}, являлся решением задачи оптимизации, необходимо и достаточно, чтобы существовали множители Лагранжа, $p_0 > 0$, $p = (p_1, \ldots, p_m) \geq 0$ и $s = (s_1, \ldots, s_n) \geq 0$, такие, что:
\begin{equation}\label{syst2}
\begin{aligned}
&(\hat{X}^j, \hat{l}^j) \in \text{Argmax}\left\{p_jF_j(X^j, l^j) - p X^j - sl^j \Biggl|X^j \geq 0, l^j \geq 0\right\},\quad j = 1,\ldots, m;\\
&p_j\left[F_j(\hat{X}^j, \hat{l}^j) - \hat{X}^0 - \sum\limits_{i=1}^{m}\hat{X}^i_j\right] = 0;\quad j=1,\ldots, m\\
&s_k\left[l_k - \sum\limits_{j=1}^m \hat{l}^j_k\right] = 0,\quad k = 1,\ldots,n;\\
&\hat{X}^0 \in \text{Argmax}\left\{p_0F_0(X^0) - pX^0\right\}.
\end{aligned}
\end{equation}

Будем интерпретировать множители Лагранжа как цены: $p$ --- вектор цен на продукцию отраслей, а $s$ --- вектор цен на первичные ресурсы. Из первого условия системы \eqref{syst2} видно, что спрос и предложение в отраслях определяется из максимизации прибыли при установившихся наборах цен $(p, s)$.

В работе \cite{Inverse_Shan} подробно описаны способы получения балансовых значений цен и формирования ВВП в рамках данной модели.


\subsubsection{Решение обратной задачи}

В данном разделе мы определим 

Введем следующие обозначения: $Z^0 = (Z^0_1,\ldots, Z^0_m)$, где $Z^0_i = \sum\limits_{i=m+1}^{m+k} X^j_i$ --- конечное потребление продукта $i$-й отрасли. Суммарную стоимость продуктов всех отраслей обозначим $A_0 = \sum\limits_{i=1}^mZ^0_i$, а сумму затрат $j$-й отрасли на производство через $A_j = \sum\limits_{i=1}^m Z^j_i,\ j=1,\ldots,m$.

Положим
\begin{itemize}
\item $a^j_i = \dfrac{Z^j_i}{A_j},\ i=1,\ldots,m,\ j=1,\ldots,m$ --- долю затрат $j$-й отрасли на продукцию $i$-й отрасли;
\item $b^j_i = \dfrac{Z^j_{m+i}}{A_j},\ i=1,\ldots,n,\ j=1,\ldots,m$ --- долю затрат на $j$-й отрасли на $m+i$-й первичный ресурс;
\item $a^0_i = \dfrac{Z^0_i}{A_0},\ i=1,\ldots,m$ --- долю $i$-й отрасли в совокупном выпуске.
\end{itemize}

В работе \cite{Inverse_Shan} показано, что при выборе в качестве производственных функций и функции полезности функции Кобба--Дугласа вида:

$$
F_i(X^i, l^i) = \alpha_i(X^i_1)^{a_1^i}\ldots (X_m^i)^{a^i_m}(l^i_1)^{b^i_1}\ldots(l_n^i)^{b^i_n},\ i=1,\ldots,m
$$
при $\alpha_0 = A_0\left[\prod\limits_{i=1}^m\dfrac{1}{Z^0_i}\right]^{a^0_i}$ и $\alpha_j = A_j\left[\prod\limits_{i=1}^m\dfrac{1}{Z^j_i}\right]^{a^j_i}\left[\prod\limits_{i=1}^n\dfrac{1}{Z^j_{m+i}}\right]^{b^j_i}$набор значений $\{\hat{X}^0_i = Z^0_i, \hat{X}^j_i=Z^j_i, l^j_{m+t}\}$ является решением задачи \eqref{syst1}, а значит, модель корректно объясняет исходные данные.

В таком случае, ВВП рассматриваемой экономики можно записать следующим образом:
$$
GDP = \dfrac{A_0}{\lambda},
$$
где $\lambda = \frac{1}{F_0(a_0)e^{\mu_1a^0_1 + \ldots + \mu_m a^0_m}}$.

Для описания шоков в модели необходимо домножить производственную функцию соответствующей отрасли на долю, характеризующую изменения.

\subsection{Агентная модель распространения COVID-19}

Рассматриваемая агентная модель описана в работе \cite{COVID_model} и реализована в среде COVASIM в репозитории \cite{COVID_code}. В рамках данной модели рассматривается группа агентов, каждый из которых имеет свой возраст, пол и принадлежность неким социальным группам и институтам. Агенты поделены на группы по возрастам (по 10 лет) и могут взаимодействовать в рамках домохозяйств, работы или учебы (в зависимости от возраста), или общественных местах. При близком контакте инфицированного агента со здоровым, с заданной вероятностью происходит заражение.
Каждый день моделируется взаимодействие агентов между собой и их состояние здоровья.

Такая модель позволяет учитывать социальную составляющую заболевания, меры предосторожности такие как самоизоляция, локдауны и дистанционное обучение.

\subsection{Модификации в моделях}

Для оценки влияния пандемии COVID-19 на экономику выбранного региона необходимо внести в модели следующие изменения.

Основным изменением является добавление в агентную модель распростанения COVID-19 новый параметр для агентов: отрасль, в которой он работает. Данная модификация позволит учитывать влияние распространения коронавирусной инфекции на отдельные отрасли с учетом их специфики. Так, в отраслях, входящих в сферу услуг заражаемость может быть выше, а ограничения типа локдауна или самоизоляции клиентов приведет к серьезным шокам, а ограничение перемещений может оказать существенное влияние на транспортную отрасль.

В качестве макроэкономических параметров, изменение которых планируется наблюдать рассматривается валовый региональный продукт (ВРП), формирующийся аналогично ВВП в модели межотраслевых балансов.

Шоки в экономической составляющей модели могут быть сформированы через уменьшение работников в отраслях (в связи с болезнью или смертью), уменьшение спроса на соответствующие товары (сфера услуг и транспорт).
\subsection{Алгоритм взаимодействия моделей}

В агентной модели распространения коронавирусной инфекции изменение состояний агентов рассчитывается каждый день.

Для определения шоков предполагается отслеживать изменение количества трудоспособных агентов в каждой из отраслей в течение 30 дней, рассматривать средневзвешенные (с течением времени, большие веса ближе к конце периода в 30 дней) значения отношения доли работников в отрасли к доковидным значениямю Далее, на основе полученных значений формировать шоки в производственных функциях. Для отдельных отраслей (таких как транспорт или сфера услуг) при расчете шоков также необходимо учитывать показатели, характеризующие стремление агентов к самоизоляции на рассматриваемом периоде в 30 дней.

В результате, для каждого периода в 30 дней рассчитываются значения шоков, а на их основе моделируются новые значения ВВП при обновленной структуре издержек. Такой подход позволяет наблюдать ежемесячные изменения в экономической ситуации в регионе. На основе полученных значений ВВП также можно построить значения ВВП на душу населения.
\section{Необходимая статистика}
Для реализации данной модели необходимы следующие статистические данные:
\begin{itemize}
\item Таблица "затраты-выпуск", есть для РФ для 2016, \cite{Rosstat_stat};
\item Таблица долей отраслей в ВРП\ по регионам, есть за 2021 год, \cite{Region_stat};
\item Распределение работников по видам деятельности внутри регионов, есть за 2021 год, \cite{Region_stat}
\item Статистика заболевших по видам деятельности, \textbf{данных нет}
\end{itemize}

\subsection{Недостающие данные}
На данном этапе работы не удалось получить данные о распределении зараженных граждан по отраслям или группам отраслей, в которых они работают.

Необходимо собрать данные о количестве заболевших в регионе по отдельным отраслям или данные о доле заболевших в конкретных отраслях внутри региона по следующим группам видов деятельности: 
\begin{itemize}
\item производство;
\item сфера услуг;
\item торговля;
\item перевозки;
\item работа в офисах;
\item строительство;
\item образование.
\end{itemize}

Полный список видов экономической деятельности приложен далее \cite{Region_stat}, более подробная статистика о заболеваемости по представленным ниже видам деятельности также представляет интерес:
\begin{itemize}
\item Сельское, лесное хозяйство, охота, рыболовство и рыбоводство;
\item Добыча полезных ископаемых;
\item Обрабатывающие производства;
\item Обеспечение электрическое энергией, газом и паром; кондиционирование воздуха;
\item Водоснабжение; водоотведение, организация сбора и утилизации отходов, деятельность по ликвидации загрязнений;
\item Строительство;
\item Торговля оптовая и розничная; ремонт автотранспортных средств и мотоциклов;
\item Транспортировка и хранение;
\item Деятельность гостиниц и предприятий общественного питания;
\item Деятельность в области информации и связи;
\item Деятельность по операциям с недвижимым имуществом;
\item Образование;
\item Деятельность в области здравоохранения и социальных услуг;
\item Другие виды деятельности.
\end{itemize}
\section{Планы и задачи}
В дальнейшем планируется построить алгоритм преобразования матрицы межотраслевого баланса для всей страны к матрице межотраслевого баланса для отдельных регионов по данным статистики о влиянии отдельных отраслей на ВРП выбранного региона.
Также необходимо ввести описанные ранее изменения в агентную модель распространения коронавирусной инфекции на основе данных о распределении агентов по отраслям внутри региона и исследованиях о заболеваемости работников в зависимости от отраслей.

\begin{thebibliography}{00}
\bibitem{Inverse_Shan}
Россоха~А.~В., Шананин~А.~А, <<Обратные задачи анализа межотраслевых балансов>>, Матем. моделирование, 33:3 (2021), 39 -- 58.
\bibitem{Duality_Shan}
Шананин~А.~А. Двойственность по Янгу и агрегирование балансов // Доклады РАН. Математика, информатика, процессы управления, 2020, т.493, с.81-85.
\bibitem{Akimova}
Акимова~Е.~Д. Выпускная квалицикационная работа <<Сетевые модели экономического роста>>, Москва, 2021.
\bibitem{Leontev}
Леонтьев~В.~В. Экономические эссе. -- М.: Политиздат, 1990, 404 с.
\bibitem{COVID_model}
 Криворотько О.И., Кабанихин С.И., Сосновская М.И., Андорная Д.В. Анализ чувствительности и идентифицируемости математических моделей распространения эпидемии COVID-19. Вавиловский журнал генетики и селекции, \textbf{25}(1), 82--91 (2021).
\bibitem{COVID_code}
COVID-19 Agent-based Simulator \underline{https://github.com/InstituteforDiseaseModeling/covasim}
\bibitem{Region_stat}
Регионы России. Социально-экономические показатели - 2021 год \underline{https://gks.ru/bgd/regl/b21\_14p/Main.htm}
\bibitem{Rosstat_stat}
Росстат, таблицы <<затраты-выпуск>> --- 2016 год \underline{https://rosstat.gov.ru/statistics/accounts}
\end{thebibliography}
\end{document} 