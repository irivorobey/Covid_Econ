\documentclass[12pt, a4paper]{article}
\usepackage{amsthm,amsfonts,amsmath,amssymb,amscd}
\usepackage[T2A]{fontenc}                         
\usepackage[utf8]{inputenc}                      
\usepackage[english, russian]{babel}
\usepackage{graphicx}
\usepackage{indentfirst}
\usepackage{multicol}
\usepackage{titlesec}
\usepackage{url}
\usepackage[hidelinks]{hyperref}
%\usepackage{cite} 
%\usepackage{psfrag}
%\IfFileExists{pscyr.sty}{\usepackage{pscyr}}{}    %

\usepackage[top=2cm,bottom=2cm,left=1.5cm,right=1.5cm]{geometry}

\titleformat{\section}
  {\normalfont\fontsize{12}{15}\bfseries}{\thesection}{1em}{}
  
\titleformat{\subsection}
  {\normalfont\fontsize{12}{14}\bfseries}{\thesubsection}{1em}{}
  
\titleformat{\subsubsection}
  {\normalfont\fontsize{12}{12}\bfseries}{\thesubsubsection}{1em}{}
 
\begin{document}
\begin{center}
Отчет о проделанной работе от 30.08.2022\\

\textbf{<<Разработка комбинированной математической модели распространения COVID-19 с учетом экономических агентов>>}\\
Воробьева Ирина
\end{center}

Для моделирования влияния пандемии COVID-19 на экономику предлагается взять за основу модели межотраслевого баланса. На основе данной модели предлагается изучать распространение шоков в экономике. В качестве основного шока предлагается рассматривать влияние пандемии Covid-19 на занятость и трудоспособность населения.

\section{Построение матрицы межотраслевого баланса для Новосибирской области}
\subsection{Используемые данные}
Использование методов анализа межотраслевого баланса предполагает наличие соответствующей матрицы. Матрицы межотраслевого баланса в Российской Федерации строятся только на национальном уровне, поэтому при моделировании экономики отдельно взятого региона необходимо перейти от национальных матриц межотраслевого баланса к региональным.

В первую очередь нужно определить, на основе каких статистических данных строится модель межотраслевого баланса. Симметричная таблица Затрат-Выпусков строится раз в 5 лет, наиболее актуальной является таблица за 2016 год \cite{RosstatZV}. Однако, в работах \cite{AkimovaCourse}, \cite{MOBLoc} при анализе используются таблицы использования товаров и услуг, которые Росстат предоставляет ежегодно. В работе \cite{AkimovaCourse} строится матрица норм затрат, усредненная за период с 2012 по 2019 год, однако в рассматриваемой задаче о влиянии пандемии на экономику региона будут рассматриваться таблицы за предшествующий пандемии 2019 год  \cite{RosstatTRI}.

Таблица использования отечественной продукции в основных ценах содержит информацию о том, каким образом выпуски отраслей распределяются по произведенной продукции. Другой важной информацией в таблице являются данные об импорте, экспорте товаров и услуг и валовой добавленной стоимости. Необходимо проверять соответствие продукции, представленной в таблице, выделенным отраслям. Сделать это можно с помощью сравнения кодов ОКВЭД и ОКПД \cite{OKPD}.
 Всего в таблице выделены 61 отрасль и 61 вид товаров.

Алгоритмы локализации матриц межотраслевого баланса для регионального уровня описаны в \cite{MOBFlegg}, а в работе \cite{MOBLoc} применены для регионов Российской Федерации по таблице использования товаров и услуг за 2017 год. В работе \cite{MOBAccur} 
 проведены исследования, позволяющие выбрать наиболее подходящий метод построения коэффициентов локализации. Для применения методов локализации используются данные о занятости населения в различных отраслях экономики как внутри рассматриваемого для локализации региона, так на национальном уровне. Такие данные ежегодно собирает и предоставляет Росстат \cite{RegionStat}. 
 Среди представленной информации есть данные о численности рабочей силы по годам и о распределении среднегодовой численности занятых по видам экономической деятельности за различные годы (до 2020).

Для уточнения полученных с помощью метода локализации результатов воспользуемся данными об отраслевой структуре валовой добавленной стоимости в рассматриваемом регионе и данными о ВДС по регионам внутри страны. Такие данные также ежегодно (с задержкой на 2 года) публикуются Росстатом \cite{RegionStat}.

В результате получены все необходимые данные для построения таблицы метожраслевого баланса для любого из регионов РФ. Далее, в качестве примера, будем рассматривать Новосибирскую область. По собранным данным аналогичные преобразования и построения можно провести для любого субъекта Федерации или федерального округа.

\subsubsection{Обработка данных}

В описанных ранее таблицах по-разному выбраны отрасли, для которых происходит агрегирование значений. Для таблицы использования товаров и услуг выделяется 61 отрасль, в таблице с распределением занятых выделено всего групп 14 отраслей. Таким образом, необходимо построить таблицу соответствия отраслей из разных таблиц и провести агрегирование по отраслям. Для группировки будем использовать оператор агрегирования $T = \{t_{ij}\}_{i = 1, \ldots, m}^{j = 1, \ldots, n}$, где $n$ --- число отраслей до агрегации, $m$ --- число отраслей в матрице после агрегации. Если мы хотим, чтобы после действия оператора $j$-я отрасль попала в $i$-ю, то $t_{ij} = 1$, если нет, то $t_{ij} = 0$. В таком случае, для матрицы технологических коэффициентов $A = \{a_{ij}\}_{i, j = 1, \ldots, n}$ агрегированная матрица $\bar{A}$ получается по формуле $\bar{A} = T A T^T.$ Также необходимо помнить о группировке значений внутри векторов использования продукции и первичных ресурсов в соответствующих отраслях. В приложении представлена таблица соотношения между отраслями из таблицы о распределении среднегодовой численности занятых по видам экономической деятельности и таблицы использования товаров и услуг, таблица, группирующая товары по соответствующим им отраслям, а также полученная в результате агрегирования отраслей усеченная таблица использования товаров и услуг для Российской Федерации за 2019 год. В этой таблице содержатся данные о межотраслевом взаимодействии 14 выделенных групп отраслей, совокупные данные об использовании продукции, а также данные о первичных ресурсах: валовая добавленная стоимость, в которую входит, в том числе, и оплата труда, и остальные объединенные внутри групп отраслей первичные ресурсы.

Для построения коэффициентов локализации строится отношение численности трудящихся в отрасли в выбранном регионе к численности трудящихся в этой отрасли на национальном уровне. Для того, чтобы сравнивать эти величины, объединим данные из двух рассматриваемых таблиц, связанных с занятостью населения. В первой таблице <<Рабочая сила>> содержатся данные о численности рабочей силы в регионе, во второй таблице <<Распределение среднегодовой численности занятых по видам экономической деятельности>> результаты представлены в процентах внутри региона. Для того, чтобы получить искомые данные умножим каждую строку второй таблицы (данные о распределении по отраслям внутри региона) на совокупную численность занятых внутри региона из первой таблицы.
Для данных о ВДС регионов проводим аналогичные действия и получаем ВДС по отраслям внутри региона в денежном выражении.

В результате, были получены таблицы межотраслевого баланса для РФ за 2019 год по группам отраслей, соответствующих данным о занятости населения. Имеющиеся данные позволяют построить локальную матрицу межотраслевого баланса для Новосибирской области.

\subsection{Построение матрицы локализации}

Для построения коэффициентов локализации воспользуемся алгоритмом из 
\cite{MOBLoc}. 
В рассматриваемом методе элементы регионального межотраслевого баланса $z_{ij}^R$ получаются из произведения регионального отраслевого выпуска $x_j^R$ и регионального технологического коэффициента $a_{ij}^R$: $$z^R_{ij} = a^R_{ij} x_j^R$$. 

Из предположения о схожей структуре производства одного продукта в различных регионах, можно получить формулу для определения $x_{j}^R$ из национального отраслевого выпуска:
$$
x_j^R = \dfrac{L_j^R}{L_j^N}x_j^N,
$$
где $L_j^R,\ L_j^N$ --- региональная и национальная занятость в отрасли $j$. Данные о национальном отраслевом выпуске содержатся в таблице использования товаров и услуг, региональная и национальная занятость является суммой по строкам в построенной ранее таблице.

Технологический коэффициент можно интерпретировать как количество единиц региональной продукции $i$ для производства единицы региональной продукции $j$. 
Коэффициент локализации $a_{ij}^R$ строится путем нормировки национальных коэффициентов: 
$$a_{ij}^R = \left\{ \begin{aligned}&t_{ij}a_{ij}^N,\quad t_{ij} \leqslant 1,\\&a_{ij}^N,\quad t_{ij} > 1\end{aligned}\right.$$. 

Коэффициент $t_{ij}$ отвечает за локализацию. Существуют различные подходы к его построению, далее будем рассматривать коэффициент FLQ, который, согласно статьям, 
\cite{MOBAccur}, \cite{MOBMonteCarlo} 
дает наиболее точные результаты. Такой метод расчета коэффициента локализации позволяет учитывать как межотраслевую структуру занятости в регионе, так и размер региона.

Введем следующие обозначения: $L^R$ --- общая занятость в регионе, $L^N$ --- общая занятость на национальном уровне, $L_i^R,\ L_i^N$ --- общая занятость в $i$-й отрасли на региональном и национальном уровне соответственно, $\delta \geqslant 1$ --- эндогенный параметр. При таких обозначениях искомый коэффициент $t_{ij}$ равен
$$
t_{ij} = \mathrm{FLQ}_{ij} = \left\{\begin{aligned}
& \dfrac{L_i^R L_j^N}{L_i^N L_j^R}\lambda,\quad i \neq j,\\
& \dfrac{L_i^R l^N}{L^R L_i^N}\lambda,\quad i=j,
\end{aligned}\right.
$$
где $\lambda = \left[\log_2\left(1 + \dfrac{L^R}{L^N}\right)\right]^\delta$, $\delta \in [0, 1]$. Множитель $\lambda$ позволяет
\cite{InverseShan}т учитывать размер региона внутри страны, а параметр $\delta$ в его определении выбирается таким образом, чтобы получить наиболее согласованные с реальными данными результатами. Параметр $\delta$ выбирался таким образом, чтобы суммарный выпуск отраслей не превышал разницу между ожидаемым совокупным выпуском и валовой добавленной стоимостью, полученной из статистических данных.

Таблица для Новосибирска за 2019 год, полученная описанным образом, приведена в приложении.

\section{Моделирование воздействия шоков на экономику Новосибирской области}
\subsection{Описание модели с учетом имеющихся данных}
В основе выбранного метода для исследования экономики Новосибирской области лежит модель межотраслевого баланса, так как именно моделирование межотраслевых связей позволяет изучать комплектное влияние шоков, в том числе в отдельных отраслях, на экономику рассматриваемой системы. Первые работы по решению задачи построения модели межотраслевого баланса написаны А.В.Леонтьевым \cite{Leontief}. В его модели считается, что связь между отраслями можно описать с помощью матрицы технологических коэффициентов: они описывают потребность $i$-й отрасли в продукции $j$-й отрасли. Считается, что произведенная продукция расходуется на следующий производственный цикл и на потребление. Обозначим потребление через $Z = (Z_1, \ldots, Z_n)$, а выпуск через $Y = (Y_1, \ldots, Y_n)$. Матрицу технологических коэффициентов обозначим через $A = \{a_{ij}\}_{i,j = 1, \ldots, n}$, причем $a_{ij} \geqslant 0$ для любых $i, j$. В таком случае, сформулированную ранее гипотезу о распределении произведенной продукции можно записать следующим образом:
$$
Z = (E - A)Y,
$$
где $E$ --- единичная матрица.

Рассмотрим систему из $m$ отраслей. Введем вектор $X^j = (X^j_1, \ldots, X^j_m)$, где $X_i^j$ --- объем продукции $i$-й отрасли, необходимый для производства в $j$-й отрасли. Будем также предполагать, что в процессе производства каждая из отраслей использует первичные ресурсы, обозначим это через $l^j = (j_1^j, \ldots, l_n^j)$. Для каждой отрасли введем производственные функции, $F_j(X^j, l^j)$, будем считать, что они обладают неоклассическими свойствами (вогнуты, монотонно неубывают, непрерывны и обращаются в нуле в нуль).

Через $X^0 = (X^0_1,\ldots, X^0_m)$ обозначим объемы поставок производимой продукции отраслями, а $F_0(X^0)$ --- функция полезности репрезентативного потребителя.

Рассмотрим задачу об оптимальном использовании первичных ресурсов, ограниченных вектором $(l_1, \ldots, l_n)$, в целях максимизации функции полезности потребителя \cite{InverseShan}, \cite{Akimova}. Запишем данную задачу в виде задачи выпуклого программирования \eqref{syst1}.

\begin{equation}\label{syst1}
\begin{aligned}
&F_0(X^0) \rightarrow \max;\\
&F_j(X^j, l^j) \geq \sum\limits_{i=0}^{m}X_j^i, j = 1,\ldots,m;\\
&\sum\limits_{j=1}^m l^j \leq l;\\
&X^0 \geq 0, \ldots, X^m \geq 0, l^1 \geq 0,\ldots,l^n \geq 0. 
\end{aligned}
\end{equation}

Далее через $Z = \{Z_i^j\}_{i=1,\ldots m+n}^{j = 1, \ldots, m+k}$ будем обозначать усеченную таблицу использования товаров и услуг. Число рассматриваемых отраслей $m = 14$, все потребители собраны в одного $k = 1$, а в качестве первичных ресурсов рассматриваются ВДС и Импорта $n = 2$.

По таблице использования товаров и услуг построим:
\begin{itemize}
\item $Z^0 = (Z_1^0, \ldots, Z_m^0)$, где $Z_i^0 = \sum\limits_{j = m + 1}^{m + k}$. В рассматриваемой нами таблице $k = 1$ (все конечные потребители объединены в одного), поэтому в рассматриваемом случае $Z^0$ - последний столбец матрицы использования товаров и услуг;
\item $A_0 = \sum\limits_{i=1}^m Z_i^0$ --- совокупное конечное потребление;
\item $A_j = \sum\limits_{i=1}^{m+n} Z_i^j$ --- совокупные средства производства, необходимые $j$-й отрасли;
\item $a_{ij} = Z_i^j / A_j$ --- технологические коэффициенты;
\item $b_{ij} = Z_{m+i}^j / A_j$ --- доля первичных ресурсов в производственных средствах $j$-й отрасли;
\item $a^0_i = Z_i^0 / A_0$ --- доля потребления, покрытая $i$-й отраслью.
\end{itemize}

В работе 
\cite{InverseShan} 
показано, что набор значений переменных из таблицы межотраслевого баланса 
$$
\left\{
\begin{aligned}
&\hat{X}_i^0 = Z_i^0,\quad i = 1,\ldots, m,\\
&\hat{X}_i^j = Z_i^j,\quad i = 1,\ldots m, j=1,\ldots,m,\\
&\hat{l}_t^j = Z_{m+t}^j, \quad t = 1, \ldots, n, j = 1,\ldots, m
\end{aligned}
\right\}
$$
является решением задачи выпуклого программирования \eqref{syst1}.

Для задачи выпуклого программирования \eqref{syst1} можно построить двойственную задачу с сопряженными переменными $p = (p_1, \ldots, p_m)$ и $s = (s_1, \ldots, s_n)$, которые интерпретируются как цены на продукцию, производимую отраслями, и цены на первичные ресурсы соответственно. Двойственная задача имеет вид \eqref{syst2}.

\begin{equation}\label{syst2}
\begin{aligned}
&(\hat{X}^j, \hat{l}^j) \in \text{Argmax}\left\{p_jF_j(X^j, l^j) - p X^j - sl^j \Biggl|X^j \geq 0, l^j \geq 0\right\},\quad j = 1,\ldots, m;\\
&p_j\left[F_j(\hat{X}^j, \hat{l}^j) - \hat{X}^0 - \sum\limits_{i=1}^{m}\hat{X}^i_j\right] = 0;\quad j=1,\ldots, m\\
&s_k\left[l_k - \sum\limits_{j=1}^m \hat{l}^j_k\right] = 0,\quad k = 1,\ldots,n;\\
&\hat{X}^0 \in \text{Argmax}\left\{p_0F_0(X^0) - pX^0\right\}.
\end{aligned}
\end{equation}

Будем рассматривать производственные функции вида Кобба--Дугласа:
$$
F_j(X^j, l^j) = \alpha_j (X1^j)^{a_{1i}}\cdot \ldots \cdot (X_m^j)^{a_{mj}}(l_1^j)^{b_{1j}}\cdot \ldots \cdot (l_n^j)^{b_{nj}}, \quad j = 1, \ldots, m.
$$

В качестве функции полезности потребителей также рассмотрим функцию Кобба--Дугласа вида 
$$
F_0(X^0) = (X^0_1)^{a_1^0}\cdot \ldots \cdot (X_m^0)^{a^0_m}.
$$

При таком выборе функций полезности и производственных функций в работе 
\cite{InverseShan} 
показано, что решением двойственной задачи для цен на продукцию отраслей при некоторых фиксированных ценах на первичные ресурсы является вектор $p = (p_1, \ldots, p_m)$, c
$$
p_j = e^{\mu_j} s_1^{c_{ij}}\cdots \ldots \cdot s_n^{c_{nj}},
$$
где\begin{itemize}
\item $C = \{c_ij\}_{i = 1,\ldots,n}^{j = 1, \ldots, m} = (E - A^T)^{-1}B^T,$ 
\item $d = (\ln(F_1(a_1, b_1)), \ldots, \ln F_m(a_m, b_m))^T,$
\item $\mu = -(E - A^T)^{-1}d$.
\end{itemize} 

В таком случае, ВВП (валовой внутренний продукт) рассматриваемой экономики можно записать следующим образом:
\begin{equation}\label{GDP}
GDP = \dfrac{A_0}{F_0(a_0)}e^{-(\mu_1a^0_{1}+\ldots+\mu_m a^0_m)}.
\end{equation}

Таким образом, мы знаем, что экономическая система находится в равновесии, и что наблюдаемые статистически значения матрицы межотраслевого баланса являются оптимальными по распределению первичных ресурсов с целью максимизации ВВП. Однако, в моменты воздействия шоков на экономику система выходит из положения равновесия. Анализ влияния шоков помогает показать, каким образом неоптимальное поведение сказывается на функции полезности потребителей, а через нее и на ВВП. 

\subsection{Описание введения шоков в экономику}
В предложенной задаче рассматривается распространение шоков, вызванное изменениями выпусков отраслей: во время пандемии многие предприятия вынуждены были приостановить работу, значительная часть сотрудников также не могла выполнять свои обязанности в связи с болезнью, в результате чего объем производимой продукции уменьшился. Рассмотрим, каким образом изменения выпуска в отдельных отраслях (более или менее подверженных воздействиям пандемии) оказывают влияние на экономику рассматриваемых экономических систем.

В основе описанной ранее модели лежит гипотеза постоянства коэффициентов прямых затрат в межотраслевом взаимодействии, которую можно описать следующим образом:
\begin{equation}\label{Leontev}
Y = AY + Z,
\end{equation}
где $Y$ --- вектор выпуска товаров, $Z$ - вектор потребления товаров, а $A = \{a_{ij}\}$ --- матрица технологических коэффициентов прямых затрат. 

При изменении выпусков изменяется величина $Y$, которая, при выполнении гипотезы о сохранении технологических коэффициентов, оказывает влияние на конечное потребление $X_0 = Z$. В результате, через объемы конечного потребления оказывается влияние на формирование значения ВВП рассматриваемой экономики по формуле \eqref{GDP}.
Сравнение значений ВВП до и после внесения шоков могут отражать силу изменений, вызванных воздействием шоков, в частности пандемии Covid-19, в рассматриваемой экономике.

\subsection{Описание построения численного моделирования}

Главным параметром, изменение которого рассматривается при моделировании является относительное изменение ВВП. В описанной ранее модели нет четкой привязки в текущему ВВП, тем более, что на региональном уровне в статистике принято выделять значение ВРП (валовой региональный продукт), которое по сути является валовой добавленной стоимостью, а не валовым внутренним продуктом. Полученные в результате моделирования значения ВВП региона сравниваются со значениями, полученными при моделировании оптимального случая (без внесения изменений в выпуск), а результат выражается в процентах от ВВП при оптимальных значениях параметров.

В первую очередь, при моделировании из матрицы использования товаров и услуг строятся коэффициенты, описанные ранее: $A_0,\ A_j,\ a_{ij},\ b_{ij}$ и прочие. На их основе строятся производственные функции и функция полезности вида Кобба--Дугласа с коэффициентами 
$$\alpha_j = A_j \prod\limits_{i=1}^{m} \dfrac{1}{(Z_i^j)^{a_{ij}}}\prod\limits_{i=1}^{n}\dfrac{1}{(Z_{m+i}^j)^{b_{ij}}},\quad \alpha_0 = A_0 \prod\limits_{i=1}^m\dfrac{1}{(Z_i^0)^{a_i^0}}.$$
Таким образом были построены функции полезности для оптимальных финансовых потоков и использования начальных запасов.

Для моделирования падения выпуска будем уменьшать оптимальные значения выпуска на 10\%, 15\%, 20\%. Затем, пользуясь матрицей технологических коэффициентов восстановим значения конечного использования продукции. Подставляя значения в описанную ранее функцию для определения ВВП и нормируя полученный результат на ВВП до внесения каких-либо изменений, получим искомую величину.
\section{Результаты численного моделирования}
В данном разделе представлены результаты моделирования воздействия шоков на различные отрасли экономики Новосибирской области. Рассматриваются следующие сценарии:
\begin{enumerate}
\item падение выпуска во всех отраслях экономики;
\item падение выпуска в трех наиболее значимых, с точки зрения ВДС, отраслях экономики;
\item падение выпуска в трех наиболее важных, с точки зрения численности занятых, отраслях.
\end{enumerate}
\subsection{Падение выпуска во всех отраслях}

Падение выпуска во всех отраслях не приводит к существенным изменениям в межотраслевом взаимодействии, так как падение одинаково для всех рассматриваемых отраслей. В этом случае относительный ВВП уменьшается пропорционально уменьшению выпусков.

Таким образом, при уменьшении выпуска на 10\%, 15\% и 20\% получим следующие результаты:
\begin{center}
\begin{tabular}{|c|c|}
\hline
Падение выпуска & Относительное изменение ВВП (\%) \\
\hline
0\% & 100\% \\
10\% &94,7\% \\
15\% & 89,5\% \\
20\% & 84,2\%\\
\hline
\end{tabular}
\end{center}

Также результаты представлены на рис.\ref{fig1}.

\begin{figure}[h]
\includegraphics[width=0.8\textwidth]{pictures/shock1.png}\label{fig1}
\caption{Изменение выпуска во всех отраслях}
\end{figure}
\subsection{Падение выпуска в главных отраслях}

Рассмотрим распределения ВДС и распределение рабочей силы по отраслям. Заметим, что в обоих случаях доминирующие отрасли (за исключением <<Других видов деятельности>>) совпадают. Эти отрасли:
\begin{enumerate}
\item транспортировка и хранение;
\item торговля оптовая и розничная;
\item обрабатывающие производства.
\end{enumerate}
Заметим, что порядки, в которых эти отрасли убывают, отличаются для двух рассматриваемых критериев. 


\begin{figure}[h]
\includegraphics[width=0.9\textwidth]{pictures/VDS.png}
\end{figure}

\begin{figure}[h]
\includegraphics[width=0.9\textwidth]{pictures/labour.png}
\end{figure}
Рассмотрим несколько вариантов изменения выпуска: изменение выпуска в отдельно выбранной отрасли из трех и изменение выпуска во всех отраслях одновременно. Это позволит выделить наиболее важную, с точки зрения влияния на ВВП экономики, отрасль и оценить относительное изменение ВВП при изменениях во всех выбранных отраслях.

Влияние падения выпуска в трех главных отраслях представлено в следующей таблице.

\begin{figure}[h]
\begin{tabular}{|c|c|c|c|}
\hline
Изменение выпуска & Обрабатывающие производства & Торговля & Транспортировка\\
\hline
0\% & 100\% & 100\% & 100\%\\
10\% & 97,5\% &98,5\% & 99,1\%\\
15\% & 96,1\%& 97,7\%& 98,5\%\\
20\% & 94,5\%& 96,8\%& 98\%\\
\hline
\end{tabular}

\end{figure}

Из полученных результатов можно сделать вывод о том, что изменения в отрасли <<Обрабатывающие производства>> являются наиболее значимыми из рассматриваемых отраслей.

При внесении изменения во все отрасли одновременно получаются следующие результаты.
\begin{center}
\begin{tabular}{|c|c|}
\hline
Падение выпуска & Относительное изменение ВВП\\
\hline
0\% & 100\%\\
10\% & 95,2\%\\
15\% & 92,7\%\\
20\% & 90,7\\
\hline
\end{tabular}
\end{center}

Получается, что уменьшение выпусков в трех отраслях одновременно приводит к более значительным изменениям ВВП, причем падение ВВП примерно равно половине от падения ВВП при уменьшении выпуска во всех отраслях экономики региона одновременно. Таким образом, выбранные отрасли являются центральными для экономики Новосибирской области.

\section{Дальнейшие исследования}
В ближайшее время планируется более подробное изучение влияния непосредственно пандемии на ситуацию на рынке труда, анализ статистики заболеваемости в регионе. Интересно выделить отрасли, которые оказались наиболее пострадавшими с точки зрения рабочей силы, и оценить масштаб ущерба (выраженный в размере шоков). Предположение о том, что выпуск отраслей уменьшается пропорционально числу заболевших в рассматриваемой отрасли может позволить построить модель, учитывающую изменение во всех рассматриваемых отраслях.
\newpage
\bibliography{sources.bib}
\bibliographystyle{plain}

%\bibitem{Inverse_Shan}
%Россоха~А.~В., Шананин~А.~А, <<Обратные задачи анализа межотраслевых балансов>>, Матем. моделирование, 33:3 (2021), 39 -- 58.
%\bibitem{Duality_Shan}
%Шананин~А.~А. Двойственность по Янгу и агрегирование балансов // Доклады РАН. Математика, информатика, процессы управления, 2020, т.493, с.81-85.
%\bibitem{Akimova}
%Акимова~Е.~Д. Выпускная квалицикационная работа <<Сетевые модели экономического роста>>, Москва, 2021.
%\bibitem{Akimova_course}
%Акимова~Е.~Д. Курсовая работа <<Использование межотраслевого баланса для решения современных проблем межотраслевых связей>>, Москва, 2022.
%\bibitem{Leontev}
%Леонтьев~В.~В. Экономические эссе. -- М.: Политиздат, 1990, 404 с.
%\bibitem{COVID_model}
% Криворотько О.И., Кабанихин С.И., Сосновская М.И., Андорная Д.В. Анализ чувствительности и идентифицируемости математических моделей распространения эпидемии COVID-19. Вавиловский журнал генетики и селекции, \textbf{25}(1), 82--91 (2021).
%\bibitem{COVID_code}
%COVID-19 Agent-based Simulator \underline{https://github.com/InstituteforDiseaseModeling/covasim}

%\bibitem{Region_stat}
%Регионы России. Социально-экономические показатели - 2021 год \underline{https://gks.ru/bgd/regl/b21\_14p/Main.htm}
%\bibitem{Rosstat_stat}
%Росстат, таблицы <<затраты-выпуск>> --- 2016 год \underline{https://rosstat.gov.ru/statistics/accounts}
%\bibitem{Rosstat_TRI}
%Таблицы ресурсов и использования товаров и услуг Российской Федерации за 2019 год \underline{https://rosstat.gov.ru/storage/mediabank/tri-2019.xlsx}

%\bibitem{MOB_stat1}
%Коган~А.~Б., <<Межотраслевой анализ экономики Новосибирской области>>, Вестник НГУЭУ, 2015.
%\bibitem{MOB_stat2}
%Котова~Т.~Е. <<Оценка внешнеторговых эффектов в экономике Хабаровского
%края на основе использования таблиц «затраты–выпуск» >> Пространственная
%экономика. 2012. No 1. С. 43–68.
%\bibitem{MOB_Flegg}
%A.~T.~Flegg , C.~D.~Webber \& M.~V.~Elliott (1995) On the Appropriate Use of Location Quotients in Generating Regional Input–Output Tables, Regional Studies, 29:6, 547-561, DOI:~10.1080/00343409512331349173
%\bibitem{MOB_Accur}
%Kronenberg~T. <<Regional input-output models and the treatment of imports in the European System of Accounts (ESA)>> Jahrbuch für Regionalwissenschaft. 2012. Vol. 32. Рр. 175-191.
%\bibitem{MOB_Monte_Carlo}
%Bonfiglio~A. <<On the Parameterization of Techniques for Representing Regional Economic Structures>> Economic System Research. 2009. Vol. 212. Pp. 115-127.
%\bibitem{MOB_loc_base}
%Пономарев~Ю.~Ю., Евдокимов~Д.~Ю., <<Построение усеченных таблиц <<Затраты-Выпуск>> для регионов России с использованием коэффициентов локализации>>, Проблемы прогнозирования, 2021, No 6.
%\bibitem{OKVED}
%Расшифровка кодов ОКВЭД и их классификация \underline{https://xn----dtbec0aczc1l.xn--p1ai/\#1}
%\bibitem{OKPD}
%Общероссийские экономические классификаторы, закрепленные за минэкономразвития России \underline{https://www.economy.gov.ru/material/departments/d18/obshcherossiyskie\_klassifikatory\_zakreplennye\_za\_minekonomrazvitiya\_rossii/}.

\newpage
\section{Приложения}
\subsection{Соотношение отраслей таблицы распределения занятых по отраслям и отраслей из таблицы использования товаров и услуг}

Далее: 
\begin{itemize}\item таблица 1 --- таблица распределения среднегодовой численности занятых по видам деятельности, \item таблица 2 --- таблица использования товаров и услуг в основных ценах.\end{itemize}

\begin{tabular}[t]{|c|p{6cm}|p{9cm}|}
\hline
	№ & Таблица 1 & Таблица 2\\
\hline
1 & Сельское, лесное хозяйство, охота, рыболовство и рыбоводство &  
 Растениеводство и животноводство, охота и предоставление соответствующих услуг в этих областях\\\cline{3-3}
 && Лесоводство и лесозаготовки\\ \cline{3-3}
 && Рыболовство и рыбоводство\\\cline{1-3}
2 & Добыча полезных ископаемых & Добыча полезных ископаемых\\\cline{1-3}
3 & Обрабатывающие производства & Производство пищевых  продуктов,  напитков, табачных изделий \\\cline{3-3}
&&Производство пищевых  продуктов,  напитков, табачных изделий  \\ \cline{3-3}
&& Обработка древесины и производство изделий из дерева и пробки, кроме мебели, производство изделий из соломки и материалов для плетения  \\ \cline{3-3}
&& Производство бумаги и бумажных изделий \\ \cline{3-3}
&& Деятельность полиграфическая и копирование носителей информации \\ \cline{3-3}
&& Производство кокса и нефтепродуктов \\ \cline{3-3}
&& Производство химических веществ и химических продуктов \\ \cline{3-3}
&& Производство лекарственных средств и материалов, применяемых в медицинских целях \\ \cline{3-3}
&& Производство резиновых и пластмассовых изделий \\ \cline{3-3}
&& Производство прочей неметаллической минеральной продукции \\ \cline{3-3}
&& Производство металлургическое \\ \cline{3-3}
&& Производство готовых металлических изделий, кроме машин и оборудования \\ \cline{3-3}
&& Производство компьютеров, электронных и оптических изделий \\ \cline{3-3}
&& Производство электрического оборудования \\ \cline{3-3}
&& Производство машин и оборудования, не включенных в другие группировки \\ \cline{3-3}
&& Производство автотранспортных средств, прицепов и полуприцепов \\ \cline{3-3}
&& Производство прочих транспортных средств и оборудования \\ \cline{3-3}
&& Производство мебели, прочих готовых изделий \\ \cline{3-3}
&& Ремонт и монтаж машин и оборудования \\ \cline{1-3}

\end{tabular}

\begin{tabular}[t]{|c|p{6cm}|p{9cm}|}
\hline
	№ & Таблица 1 & Таблица 2\\
\hline
4 & Обеспечение электрическое энергией, газом и паром; кондиционирование воздуха & Обеспечение электрической энергией, газом и паром; кондиционирование воздуха\\ \cline{1-3}
5 & Водоснабжение; водоотведение, организация сбора и утилизации отходов, деятельность по ликвидации загрязнений & Забор, очистка и распределение воды\\ \cline{3-3}
&& Сбор и обработка сточных вод; сбор, обработка и утилизация отходов; обработка вторичного сырья; предоставление услуг в области ликвидации последствий загрязнений и прочих услуг, связанных с удалением отходов \\ \cline{1-3}
6 & Строительство & Строительство\\ \cline{1-3}
7 & Торговля оптовая и розничная; ремонт автотранспортных средств и мотоциклов & Торговля оптовая и розничная автотранспортными средствами и мотоциклами и их ремонт\\ \cline{3-3}
&& Торговля оптовая,  кроме оптовой торговли автотранспортными средствами и мотоциклами \\ \cline{3-3}
&& Торговля розничная, кроме торговли автотранспортными средствами и мотоциклами \\ \cline{1-3}
8 & Транспортировка и хранение & Деятельность сухопутного и трубопроводного транспорта\\ \cline{3-3}
&& Деятельность водного транспорта \\ \cline{3-3}
&& Деятельность воздушного и космического транспорта\\ \cline{3-3}
&& Складское хозяйство и вспомогательная транспортная деятельность \\ \cline{3-3}
&& Деятельность почтовой связи и курьерская деятельность \\ \cline{1-3}
9 & Деятельность гостиниц и предприятий общественного питания & Деятельность гостиниц и предприятий общественного питания\\ \cline{1-3}
10 & Деятельность в области информации и связи & Деятельность издательская\\ \cline{3-3}
&& Производство кинофильмов, видеофильмов и телевизионных программ, издание звукозаписей и нот; деятельность в области телевизионного и радиовещания \\ \cline{3-3}
&& Деятельность в сфере телекоммуникаций \\ \cline{3-3}
&& Разработка компьютерного программного обеспечения, консультационные услуги в данной области и другие сопутствующие услуги; деятельность в области информационных технологий \\ \cline{1-3}
11 & Деятельность по операциям с недвижимым имуществом & Деятельность по операциям с недвижимым имуществом\\ \cline{1-3}
12 & Образование & Образование\\ \cline{1-3}
\end{tabular}


\begin{tabular}[t]{|c|p{6cm}|p{9cm}|}
\hline
	№ & Таблица 1 & Таблица 2\\
\hline
13 & Деятельность в области здравоохранения и социальных услуг & Деятельность в области здравоохранения \\ \cline{3-3}
&& Деятельность по уходу с обеспечением проживания; предоставление социальных услуг без обеспечения проживания \\ \cline{1-3}
14 & Другие виды деятельности & Деятельность финансовая и страховая\\ \cline{3-3}
&& Деятельность в области права и бухгалтерского учета; деятельность головных офисов; консультирование по вопросам управления \\ \cline{3-3}
&& Деятельность в области архитектуры и инженерно-технического проектирования; технических испытаний, исследований и анализа \\ \cline{3-3}
&&Научные исследования и разработки \\ \cline{3-3}
&& Деятельность рекламная и исследование конъюнктуры рынка \\ \cline{3-3}
&& Деятельность профессиональная научная и техническая прочая; деятельность ветеринарная  \\ \cline{3-3}
&& Аренда и лизинг \\ \cline{3-3}
&& Деятельность по трудоустройству и подбору персонала \\ \cline{3-3}
&& Деятельность туристических агентств и прочих организаций, предоставляющих услуги в сфере туризма \\ \cline{3-3}
&& Деятельность по обеспечению безопасности и  проведению расследований, обслуживанию зданий и территорий, административно-хозяйственная, вспомогательная деятельность по обеспечению функционирования организации, деятельность по предоставлению прочих вспомогательных услуг для бизнеса \\ \cline{3-3}
&& Государственное управление и обеспечение военной безопасности; социальное обеспечение \\ \cline{3-3}
&& Деятельность творческая, в области искусства и организации развлечений, библиотек, архивов, музеев и прочих объектов культуры, по организации и проведению азартных игр и заключению пари, по организации и проведению лотерей \\ \cline{3-3}
&& Деятельность в области спорта, отдыха и развлечений \\ \cline{3-3}
&& Деятельность общественных организаций \\ \cline{3-3}
&& Ремонт компьютеров, предметов личного потребления и хозяйственно-бытового назначения \\ \cline{3-3}
&& Деятельность по предоставлению прочих персональных услуг \\ \cline{1-3}

\end{tabular}

\subsection{Соотношение отраслей таблицы распределения занятых по отраслям и продукции из таблицы использования товаров и услуг}

Далее: 
\begin{itemize}\item таблица 1 --- таблица распределения среднегодовой численности занятых по видам деятельности, \item таблица 2 --- таблица использования товаров и услуг в основных ценах.\end{itemize}

\begin{tabular}[t]{|c|p{6cm}|p{9cm}|}
\hline
	№ & Таблица 1 & Таблица 2\\
\hline
1 & Сельское, лесное хозяйство, охота, рыболовство и рыбоводство &  
 Продукция и услуги сельского хозяйства и охоты\\\cline{3-3}
 && Продукция лесоводства, лесозаготовок и связанные с этим услуги\\ \cline{3-3}
 && Рыба и прочая продукция рыболовства и рыбоводства; услуги, связанные с рыболовством и рыбоводством\\\cline{1-3}
2 & Добыча полезных ископаемых & Продукция горнодобывающих производств\\\cline{1-3}
3 & Обрабатывающие производства & Продукты пищевые, напитки, изделия табачные \\\cline{3-3}
&&Текстиль и изделия текстильные, одежда, кожа и изделия из кожи  \\ \cline{3-3}
&& Древесина и изделия из дерева и пробки, кроме мебели; изделия из соломки и материалов для плетения  \\ \cline{3-3}
&&Бумага и изделия из бумаги \\ \cline{3-3}
&& Услуги печатные и услуги по копированию звуко- и видеозаписей, а также программных средств \\ \cline{3-3}
&&Кокс и нефтепродукты \\ \cline{3-3}
&& Вещества химические и продукты химические \\ \cline{3-3}
&&Средства лекарственные и материалы, применяемые в медицинских целях \\ \cline{3-3}
&& Изделия резиновые и пластмассовые \\ \cline{3-3}
&& Продукты минеральные неметаллические прочие \\ \cline{3-3}
&& Металлы основные \\ \cline{3-3}
&& Изделия металлические готовые, кроме машин и оборудования \\ \cline{3-3}
&& Оборудование компьютерное, электронное и оптическое \\ \cline{3-3}
&& Оборудование электрическое \\ \cline{3-3}
&& Машины и оборудование, не включенные в другие группировки \\ \cline{3-3}
&& Средства автотранспортные, прицепы и полуприцепы \\ \cline{3-3}
&& Средства транспортные и оборудование, прочие \\ \cline{3-3}
&& Мебель, изделия готовые прочие\\ \cline{3-3}
&& Услуги по ремонту и монтажу машин и оборудования \\ \cline{1-3}

\end{tabular}

\begin{tabular}[t]{|c|p{6cm}|p{9cm}|}
\hline
	№ & Таблица 1 & Таблица 2\\
\hline
4 & Обеспечение электрическое энергией, газом и паром; кондиционирование воздуха & Электроэнергия, газ, пар и кондиционирование воздуха\\ \cline{1-3}
5 & Водоснабжение; водоотведение, организация сбора и утилизации отходов, деятельность по ликвидации загрязнений & Вода природная; услуги по очистке воды и водоснабжению\\ \cline{3-3}
&& Услуги по водоотведению; шлам сточных вод; услуги по сбору, обработке и удалению отходов; услуги по утилизации отходов; услуги по рекультивации и прочие услуги по утилизации отходов \\ \cline{1-3}
6 & Строительство & Сооружения и строительные работы\\ \cline{1-3}
7 & Торговля оптовая и розничная; ремонт автотранспортных средств и мотоциклов & Услуги по оптовой и розничной торговле и услуги по ремонту автотранспортных средств и мотоциклов\\ \cline{3-3}
&& Услуги по оптовой торговле, кроме оптовой торговли автотранспортными средствами и мотоциклами \\ \cline{3-3}
&& Услуги по розничной торговле, кроме розничной торговли автотранспортными средствами и мотоциклами \\ \cline{1-3}
8 & Транспортировка и хранение & Услуги сухопутного и трубопроводного транспорта\\ \cline{3-3}
&& Услуги водного транспорта \\ \cline{3-3}
&&Услуги воздушного и космического транспорта\\ \cline{3-3}
&&Услуги по складированию и вспомогательные транспортные услуги \\ \cline{3-3}
&& Услуги почтовой связи и услуги курьерские\\ \cline{1-3}
9 & Деятельность гостиниц и предприятий общественного питания & Услуги гостиничного хозяйства и общественного питания\\ \cline{1-3}
10 & Деятельность в области информации и связи & Услуги издательские\\ \cline{3-3}
&& Услуги по производству кинофильмов, видеофильмов и телевизионных программ, звукозаписей и изданию музыкальных записей; услуги в области теле- и радиовещания \\ \cline{3-3}
&& Услуги телекоммуникационные \\ \cline{3-3}
&& Продукты программные и услуги по разработке программного обеспечения; консультационные и аналогичные услуги в области информационных технологий; услуги в области информационных технологий \\ \cline{1-3}
11 & Деятельность по операциям с недвижимым имуществом & Услуги, связанные с недвижимым имуществом \\ \cline{1-3}
12 & Образование & Услуги в области образования\\ \cline{1-3}
\end{tabular}


\begin{tabular}[t]{|c|p{6cm}|p{9cm}|}
\hline
	№ & Таблица 1 & Таблица 2\\
\hline
13 & Деятельность в области здравоохранения и социальных услуг & Услуги в области здравоохранения\\ \cline{3-3}
&& Услуги по предоставлению ухода с обеспечением проживания; услуги социальные без обеспечения проживания \\ \cline{1-3}
14 & Другие виды деятельности & Услуги финансовые и страховые\\ \cline{3-3}
&& Услуги юридические и бухгалтерские; услуги головных офисов; услуги консультативные в области управления предприятием \\ \cline{3-3}
&& Услуги в области архитектуры и инженерно-технического проектирования, технических испытаний, исследований и анализа\\ \cline{3-3}
&&Услуги и работы, связанные с научными исследованиями и экспериментальными разработками \\ \cline{3-3}
&& Услуги рекламные и услуги по исследованию конъюнктуры рынка \\ \cline{3-3}
&& Услуги профессиональные, научные и технические, прочие; услуги ветеринарные \\ \cline{3-3}
&& Услуги по аренде и лизингу \\ \cline{3-3}
&& Услуги по трудоустройству и подбору персонала \\ \cline{3-3}
&& Услуги туристических агентств, туроператоров и прочие услуги по бронированию и сопутствующие им услуги \\ \cline{3-3}
&& Услуги по обеспечению безопасности и проведению расследований; услуги по обслуживанию зданий и территорий; услуги в области административного, хозяйственного и прочего вспомогательного обслуживания \\ \cline{3-3}
&& Услуги в сфере государственного управления и обеспечения военной безопасности; услуги по обязательному социальному обеспечению \\ \cline{3-3}
&&Услуги в области творчества, искусства и развлечений; услуги библиотек, архивов, музеев и прочие услуги в области культуры; услуги по организации и проведению азартных игр и заключению пари, лотерей \\ \cline{3-3}
&& Услуги, связанные со спортом, и услуги по организации развлечений и отдыха \\ \cline{3-3}
&& Услуги общественных организаций \\ \cline{3-3}
&& Услуги по ремонту компьютеров, предметов личного потребления и бытовых товаров \\ \cline{3-3}
&& Услуги персональные прочие \\ \cline{1-3}

\end{tabular}


\subsection{Усеченная таблица использования товаров и услуг для Российской Федерации за 2019 год}

\begin{tabular}{|l|rrrrrr}
\hline
 &          1 &           2 &           3 &          4 &         5 &          6  \\
\hline
1               &  1287390.0 &        92.0 &   2816002.0 &     3724.0 &      60.0 &    15028.0 \\
2               &     3398.0 &   1111753.0 &   5750397.0 &   882327.0 &    2799.0 &   174001.0 \\
3               &  1141376.0 &    883845.0 &  12487483.0 &   452105.0 &  262727.0 &  3317815.0\\
4               &   139192.0 &    389844.0 &   1514335.0 &  3194692.0 &   85950.0 &    78011.0 \\
5               &     5864.0 &     14228.0 &    446466.0 &    60468.0 &  173868.0 &    16128.0 \\
6               &    22575.0 &    285736.0 &    228397.0 &   117016.0 &   26828.0 &   368356.0 \\
7               &   292351.0 &    196230.0 &   3000262.0 &   744150.0 &   81356.0 &   774626.0\\
8               &   152337.0 &   1133614.0 &   2689110.0 &    77345.0 &   47154.0 &   332401.0\\
9               &     1156.0 &      8363.0 &     30813.0 &     5514.0 &     489.0 &    16976.0 \\
10              &     7009.0 &     29166.0 &    218228.0 &    50409.0 &    4149.0 &    38361.0\\
11              &    20740.0 &    353925.0 &    328211.0 &    85572.0 &   16705.0 &   111274.0\\
12              &      716.0 &      3949.0 &     13088.0 &     3749.0 &     376.0 &     3323.0 \\
13              &     1107.0 &      4519.0 &     10561.0 &     2868.0 &     654.0 &     2128.0 \\
14              &   216633.0 &    504233.0 &   1862500.0 &   306019.0 &   64265.0 &   892858.0 \\
ВДС             &  3869506.0 &  12622497.0 &  14215328.0 &  2564061.0 &  497082.0 &  5340644.0 \\
Остальное &   389767.0 &    367612.0 &   5754452.0 &   169053.0 &   90952.0 &  1073868.0 \\
\end{tabular}

\begin{tabular}{|l|rrrrrrrr|}
\hline
& 7 &          8 &         9 &         10 &         11 &         12 &      \\
\hline
1 & 8461.0 &     7443.0 &   48795.0 &       57.0 &      429.0 &     7908.0 \\
2 &   269490.0 &   126923.0 &     119.0 &       70.0 &     1227.0 &      558.0 \\
3&    655179.0 &  1859379.0 &  347323.0 &   214216.0 &   331331.0 &   111813.0\\
4&200326.0 &   416778.0 &   42645.0 &    56480.0 &   372180.0 &   162520.0 \\
5&32457.0 &    16220.0 &    6559.0 &     4783.0 &    87396.0 &    18571.0 \\
6&59543.0 &   187963.0 &   16997.0 &    12334.0 &   395515.0 &   163337.0\\
7&624948.0 &   448149.0 &   94369.0 &    64119.0 &   145224.0 &    36548.0 \\
8&3058146.0 &  2530136.0 &   21620.0 &    50255.0 &    23045.0 &    19529.0\\
9&22312.0 &    21206.0 &    4706.0 &     8771.0 &     1294.0 &    32813.0  \\
10&  248872.0 &   104385.0 &    9797.0 &  1105923.0 &    44833.0 &    55916.0\\
11&1869811.0 &   790852.0 &  182723.0 &   270876.0 &   935145.0 &    30485.0 \\
12&7814.0 &    15145.0 &     384.0 &     3203.0 &      691.0 &    58643.0 \\
13&6956.0 &     9347.0 &    1472.0 &     2116.0 &      512.0 &    15730.0 \\
14&1794620.0 &   969039.0 &  157293.0 &   381473.0 &   396653.0 &   109080.0 \\
ВДС&  12737849.0 &  6743364.0 &  919199.0 &  2607431.0 &  9633892.0 &  3279619.0 \\
Остальное& 769006.0 &   977487.0 &  145883.0 &   417072.0 &   243401.0 &   181547.0  \\
\end{tabular}

\begin{tabular}{|l|rrr|}
\hline
& 13 &          14 &  Использование \\
\hline
1 &  6325.0 &     55289.0 &    3294114.0 \\
2 &   885.0 &     16274.0 &    9569385.0 \\
3&   336824.0 &   1478755.0 &   27485462.0 \\
4&   86142.0 &    420239.0 &    1559738.0 \\
5&   15078.0 &    109211.0 &     348117.0 \\
6&  101280.0 &    740550.0 &    9829371.0 \\
7&  197920.0 &    579167.0 &   15086371.0 \\
8& 40736.0 &    740527.0 &    4307861.0 \\
9&   37719.0 &    197245.0 &    1610507.0 \\
10& 26550.0 &    934535.0 &    2321046.0 \\
11& 59276.0 &    713288.0 &    6843885.0 \\
12&  11403.0 &     55441.0 &    4106692.0 \\
13&   41911.0 &     99147.0 &    4432913.0 \\
14& 161831.0 &   3746455.0 &   24184606.0 \\
ВДС& 2981095.0 &  23456688.0 &         \\
Остальное&  526966.0 &   2404747.0 &         \\
\hline
\end{tabular}


\subsection{Усеченная матрица использования товаров и услуг для Новосибирской области за 2019 год}


\begin{tabular}{|l|lllllll}
\hline
{} & 1 & 2 & 3 & 4 & 5 & 6 & 7  \\
\hline
1 & 11888.5 & 0.545273 &  29979.7 &  39.6464 &                                  0.638771 &  159.991 & 90.0773  \\
2  &   13.0591 & 1337.88 & 22099.8 &  3390.94 &                                            10.7571 & 668.717 &  1035.7   \\
3                      &                                            13535.5 &                    5238.44 &                      143086 &                                            5361.49 &                                            3115.67 &        39345.8 &                                            7769.74 \\
4 &                                             1414.1 &                    2310.56 &                     15384.7 &                                            26865.2 &                                            873.197 &        792.542 &                                            2035.18  \\
5 &                                            57.6932 &                    84.3277 &                     4392.57 &                                            594.917 &                                            1371.23 &        158.676 &                                             319.33 \\
6 &                                            206.681 &                    1693.52 &                     2091.05 &                                            1071.32 &                                            245.619 &         2515.6 &                                            545.135 \\
7 &                                               3936 &                    1163.03 &                     40393.3 &                                            10018.7 &                                            1095.32 &          10429 &                                            9229.38 \\
8  &                                            2046.02 &                     6718.8 &                     36117.2 &                                            1038.81 &                                            633.321 &        4464.45 &                                            41073.7 \\
9 &                                            12.5107 &                    49.5665 &                      333.47 &                                            59.6747 &                                            5.29215 &        183.721 &                                            241.469 \\
10     &                                            110.826 &                    172.863 &                     3450.61 &                                            789.787 &                                             62.952 &        541.625 &                                            3935.15\\
11 &                                            330.639 &                    2097.67 &                     5232.37 &                                            1340.71 &                                            253.462 &        1571.09 &                                            29808.7  \\
12                              &                                             9.3848 &                    23.4053 &                     171.548 &                                            49.1391 &                                            4.92833 &        43.5554 &                                             102.42 \\
13 &                                            13.3945 &                    26.7836 &                     127.786 &                                            34.7023 &                                            7.91328 &        25.7484 &                                            84.1663 \\
14                          &                                             2843.8 &                    2988.53 &                     24449.5 &                                            4017.19 &                                            843.624 &        11720.8 &                                            23558.5  \\
ВДС                                                &                                            53549.3 &                      43685 &                      181786 &                                            33820.6 &                                            8455.15 &        54958.5 &                                             209970 \\
Остальное                                          &                                            34009.5 &                    38557.2 &                      430314 &                                              48114 &                                            3586.38 &        49697.4 &                                             134578 \\
\hline
\end{tabular}


\begin{tabular}{|l|llllllll|}
\hline
{} &  8 & 9 & 10 & 11 & 12 & 13 & 14 & Использование \\
\hline
1 &    79.2395 &                                             519.48 &                                  0.606832 &                                           4.56721 &        84.19 &                                            67.3371 &                  588.617 &       80473.8 \\
2   &                    487.788 &                                           0.457338 &                                  0.269023 &                                           4.71558 &      2.14449 &                                            3.40121 &                  62.5439 &       77029.9 \\
3 &                    22050.3 &                                            4118.89 &                                   2540.38 &                                           3929.24 &      1325.99 &                                            3994.38 &                  17536.5 &        666461 \\
4  &                     4234.2 &                                            433.246 &                                   573.801 &                                           3781.11 &       1651.1 &                                            875.148 &                  4269.36 &       71113.4 \\
5  &                    159.581 &                                             64.531 &                                   47.0577 &                                           859.849 &      182.712 &                                            148.346 &                  1074.48 &       11050.2 \\
6  &                    1720.86 &                                            155.613 &                                   112.922 &                                           3621.06 &       1495.4 &                                             927.25 &                  6779.96 &        154095 \\
7&                    6033.55 &                                            1270.51 &                                   863.251 &                                           1955.19 &      492.055 &                                            2664.65 &                  7797.48 &        367035 \\
8 &                    37186.2 &                                            290.376 &                                    674.97 &                                           309.515 &      262.292 &                                            547.121 &                  9945.95 &        174021 \\
9 &                      229.5 &                                            44.9083 &                                   94.9232 &                                           14.0042 &      355.115 &                                             408.21 &                  2134.66 &       29211.2 \\
10 &                    1650.53 &                                             154.91 &                                     22528 &                                           708.897 &       884.14 &                                            419.807 &                  14776.8 &       76594.4 \\
11  &                    12607.8 &                                            2912.98 &                                   4318.33 &                                           19364.1 &      485.994 &                                            944.983 &                  11371.3 &        217452 \\
12   &                     198.51 &                                            5.03319 &                                   41.9826 &                                           9.05712 &      820.859 &                                            149.462 &                   726.68 &       84252.4 \\
13  &                    113.097 &                                            17.8109 &                                   25.6032 &                                           6.19511 &       190.33 &                                            499.935 &                  1199.66 &       84059.4 \\
14    &                    12720.8 &                                            2064.83 &                                    5007.7 &                                           5206.97 &      1431.92 &                                             2124.4 &                  52601.4 &        572116 \\
ВДС  &                     215606 &                                            15501.1 &                                   54958.5 &                                            176149 &      54958.5 &                                            63413.6 &                   242381 &              \\
Остальное   &                    250.827 &                                            5823.57 &                                   34993.1 &                                           94168.8 &      21985.6 &                                            9244.49 &                   350450 &              \\
\hline
\end{tabular}

\end{document} 


\section{Модель распространения шоков в экономике}
Данная модель уходит корнями к теории межотраслевых балансов Леонтьева \cite{Leontev}. В рамках данной модели предполагалось постоянство норм затрат на выпуск продукции в процессе межотраслевого взаимодействия. Со временем, теория Леонтьева потеряла свою актуальность в связи с возросшей взаимозаменяемостью товаров и услуг. Однако идея изучения взаимосвязей между отраслями не потеряла своей популярности, а гипотеза постоянства норм затрат преобразовалась в предположение о постоянстве структуры финансовых затрат в процессе производства товаров и услуг с учетом их отраслевой дифференциации. При использовании данного подхода рассматриваются производственные функции Кобба--Дугласа. В работах \cite{Inverse_Shan}, \cite{Duality_Shan} рассматриваются вопросы, посвященные поиску экономического равновесия в системах, описываемых матрицами межотраслевых затрат. В работе \cite{Akimova} проводится выявление отраслей--драйверов для экономики Российской Федерации и их влияние на ВВП страны.


\subsection{Таблица выпусков и затрат}

Рассмотрим матрицу $Z = \{Z^j_i\}_{i=1,\ldots,m+k}^{j=1,\ldots,m+n}$ --- матрицу межотраслевого взаимодействия, где $Z^j_i$ при $i = 1,\ldots,m,\ j=1,\ldots,m$ --- денежная сумма, полученная $i$-й отраслью от $j$-й отрасли за выполнение работы, $m$ --- количество отраслей, $n$ --- количество первичных ресурсов, $k$ --- количество конечных потребителей.

Обратной задачей в данном случае является построение модели распределения ресурсов в виде задачи выпуклого программирования, решение которой воспроизводит такую матрицу.

\subsection{Модель оптимального распределения ресурсов}
Рассмотрим группу из $m$ чистых отраслей, связанных через производственные факторы (ПФ). Обозначим $X_i^j$ объем  продукции $i$-й отрасли, необходимый для производства в $j$-й отрасли. Тогда $X^j = (X^j_1,\ldots X_m^j)$ --- затраты $j$-й отрасли на производство. Обозначим через $l^j = (l_1^j, \ldots , j_n^j)$ --- вектор первичных ресурсов, используемых $j$-й отраслью, всего первичных ресурсов $n$ видов.

Производственные функции отраслей $F_j(X^j, l^j)$ зависят от используемых в отрасли производственных факторов и первичных ресурсов. Будем считать, что производственные функции удовлетворяют неоклассическим предположениям: являются вогнутыми, монотонно неубывающими, непрерывными на $\mathbb{R}^{m+n}_+$ и равными нулю в нуле.

Внешние потребители получают объемы поставок $X^0 = (X_1^0,\ldots,X_m^0)$ от всех отраслей и имеют функцию полезности $F_0(X^0)$ также удовлетворяющую неоклассическим предположениям.

Тогда, можно поставить следующую задачу выпуклой оптимизации:

\begin{equation}\label{syst1}
\begin{aligned}
&F_0(X^0) \rightarrow \max;\\
&F_j(X^j, l^j) \geq \sum\limits_{i=0}^{m}X_j^i, j = 1,\ldots,m;\\
&\sum\limits_{j=1}^m l^j \leq l;\\
&X^0 \geq 0, \ldots, X^m \geq 0, l^1 \geq 0,\ldots,l^n \geq 0. 
\end{aligned}
\end{equation}

Для того, чтобы набор векторов $\{\hat{X}^0, \hat{X}^1, \ldots, \hat{X}^m, \hat{l}^1, \ldots, \hat{l^n}\}$, удовлетворяющий ограничениям \eqref{syst1}, являлся решением задачи оптимизации, необходимо и достаточно, чтобы существовали множители Лагранжа, $p_0 > 0$, $p = (p_1, \ldots, p_m) \geq 0$ и $s = (s_1, \ldots, s_n) \geq 0$, такие, что:
\begin{equation}\label{syst2}
\begin{aligned}
&(\hat{X}^j, \hat{l}^j) \in \text{Argmax}\left\{p_jF_j(X^j, l^j) - p X^j - sl^j \Biggl|X^j \geq 0, l^j \geq 0\right\},\quad j = 1,\ldots, m;\\
&p_j\left[F_j(\hat{X}^j, \hat{l}^j) - \hat{X}^0 - \sum\limits_{i=1}^{m}\hat{X}^i_j\right] = 0;\quad j=1,\ldots, m\\
&s_k\left[l_k - \sum\limits_{j=1}^m \hat{l}^j_k\right] = 0,\quad k = 1,\ldots,n;\\
&\hat{X}^0 \in \text{Argmax}\left\{p_0F_0(X^0) - pX^0\right\}.
\end{aligned}
\end{equation}

Будем интерпретировать множители Лагранжа как цены: $p$ --- вектор цен на продукцию отраслей, а $s$ --- вектор цен на первичные ресурсы. Из первого условия системы \eqref{syst2} видно, что спрос и предложение в отраслях определяется из максимизации прибыли при установившихся наборах цен $(p, s)$.

В работе \cite{Inverse_Shan} подробно описаны способы получения балансовых значений цен и формирования ВВП в рамках данной модели.


\subsubsection{Решение обратной задачи}


Введем следующие обозначения: $Z^0 = (Z^0_1,\ldots, Z^0_m)$, где $Z^0_i = \sum\limits_{i=m+1}^{m+k} X^j_i$ --- конечное потребление продукта $i$-й отрасли. Суммарную стоимость продуктов всех отраслей обозначим $A_0 = \sum\limits_{i=1}^mZ^0_i$, а сумму затрат $j$-й отрасли на производство через $A_j = \sum\limits_{i=1}^m Z^j_i,\ j=1,\ldots,m$.

Положим
\begin{itemize}
\item $a^j_i = \dfrac{Z^j_i}{A_j},\ i=1,\ldots,m,\ j=1,\ldots,m$ --- долю затрат $j$-й отрасли на продукцию $i$-й отрасли;
\item $b^j_i = \dfrac{Z^j_{m+i}}{A_j},\ i=1,\ldots,n,\ j=1,\ldots,m$ --- долю затрат на $j$-й отрасли на $m+i$-й первичный ресурс;
\item $a^0_i = \dfrac{Z^0_i}{A_0},\ i=1,\ldots,m$ --- долю $i$-й отрасли в совокупном выпуске.
\end{itemize}

В работе \cite{Inverse_Shan} показано, что при выборе в качестве производственных функций и функции полезности функции Кобба--Дугласа вида:

$$
F_i(X^i, l^i) = \alpha_i(X^i_1)^{a_1^i}\ldots (X_m^i)^{a^i_m}(l^i_1)^{b^i_1}\ldots(l_n^i)^{b^i_n},\ i=1,\ldots,m
$$
при $\alpha_0 = A_0\left[\prod\limits_{i=1}^m\dfrac{1}{Z^0_i}\right]^{a^0_i}$ и $\alpha_j = A_j\left[\prod\limits_{i=1}^m\dfrac{1}{Z^j_i}\right]^{a^j_i}\left[\prod\limits_{i=1}^n\dfrac{1}{Z^j_{m+i}}\right]^{b^j_i}$набор значений $\{\hat{X}^0_i = Z^0_i, \hat{X}^j_i=Z^j_i, l^j_{m+t}\}$ является решением задачи \eqref{syst1}, а значит, модель корректно объясняет исходные данные.

В таком случае, ВВП рассматриваемой экономики можно записать следующим образом:
\begin{equation}\label{GDP}
GDP = \dfrac{A_0}{\lambda},
\end{equation}
где $\lambda = \frac{1}{F_0(a_0)e^{\mu_1a^0_1 + \ldots + \mu_m a^0_m}}$.

\subsection{Распространение шоков}

В предложенной задаче рассматривается распространение шоков, вызванное изменениями выпусков отраслей: во время пандемии многие предприятия вынуждены были приостановить работу, значительная часть сотрудников также не могла выполнять свои обязанности в связи с болезнью, в результате чего объем производимой продукции уменьшился. Интересно изучить, каким образом изменения выпуска в отдельных отраслях (более или менее подверженных воздействиям пандемии) оказывают влияние на экономику рассматриваемых экономических систем.

В основе описанной ранее модели лежит гипотеза постоянства коэффициентов прямых затрат в межотраслевом взаимодействии, которую можно описать следующим образом:
\begin{equation}\label{Leontev}
Y = AY + Z,
\end{equation}
где $Y$ --- вектор выпуска товаров, $Z$ - вектор потребления товаров, а $A = \{a_{ij}\}$ --- матрица технологических коэффициентов прямых затрат. 

Данное уравнение можно интерпретировать следующим образом: выпуски различных отраслей идут на продолжение производственных циклов (первое слагаемое) и потребление (второе слагаемое). Таким образом, через это уравнение устанавливается связь между объемами производств в различных отраслях и объемами потребления.

В разделе <<Модель оптимального распределения ресурсов>> были введены функция полезности, и производственные функции отраслей (функции Кобба--Дугласа). 

Считается, что к моменту начала шоков, экономика находится в равновесном положении, то есть предприятия работают оптимально в рамках рассматриваемой модели, а текущие объемы производств являются решениями задачи выпуклого программирования \eqref{syst1}. Тогда текущие объемы производств задают коэффициенты $a_{ij},\ b_{ij},\ a_{i0}$ и $\alpha_j$ в функциях Кобба-Дугласа.

При изменении выпусков изменяется величина $Y$, которая, при выполнении гипотезы о сохранении технологических коэффициентов, оказывает влияние на конечное потребление $X_0 = Z$. В результате, через объемы конечного потребления оказывается влияние на формирование значения ВВП рассматриваемой экономики по формуле \eqref{GDP}.
Сравнение значений ВВП до и после внесения шоков могут отражать силу изменений, вызванных воздействием шоков, в частности пандемии Covid-19, в рассматриваемой экономике.
\section{Применение модели распространения шоков к Новосибирской области}
Матрицы затрат-выпусков строятся на национальном уровне раз в 5 лет, а на региональном уровне не строятся совсем. Таким образом, для того, чтобы моделировать влияние пандемии Covid-19 на отдельно взятый регион, в первую очередь, необходимо провести локализацию матрицы межотраслевых затрат.
\subsection{Региональные таблицы затрат и выпусков}


Вопросы разработки моделей межотраслевого баланса (МОБ) на региональном уровне активно разрабатывается в различных странах мира. Среди подходов к решению данной задачи можно выделить два наиболее популярных:
\begin{itemize}
\item построение матрицы межотраслевого баланса на основе статистических данных;
\item построение матрицы межотраслевого баланса с помощью мультипликаторов, приводящих матрицы Росстата к региональным матрицам.
\end{itemize}

Первый подход опирается на открытые статистические данные Росстата по регионам и на закрытые данные в форме <<1-предприятие>>. Результаты применения такого подхода описаны, например, описан в статьях \cite{MOB_stat1} и \cite{MOB_stat2}. В форме <<1-предприятие>> средние и большие компании предоставляют данные о структуре своих расходов. Данные агрегируются по отраслям и формируют матрицу межотраслевого взаимодействия. Такой метод имеет следующие недостатки:
\begin{itemize}
\item в соответствии с российским законодательством органы государственной статистики не вправе публиковать информацию о работе той или иной отрасли в случае, если отчетность подают менее трех организаций (такие случаи встречаются, причем разные в разных регионах);
\item в форме <<1-предприятие>> присутствуют недостаточно детализированные статьи расходов, что вынуждает прибегать к федеральной модели;
\item для применения модели ко множеству регионов необходимо вручную обрабатывать большие объемы информации, запрашивать доступ к платным формам <<1-предприятие>>, учитывать п.1 из приведенных ранее для каждого отдельного региона.
\end{itemize}

Второй подход предполагает использование коэффициентов локализации (Location quotient - LQ), позволяющих приводить федеральную матрицу межотраслевого баланса к региональным. Коэффициенты локализации формируются из отношения занятости в отрасли на региональном и национальном уровне. Таким образом, коэффициенты позволяют учитывать долю федеральных производственных средств, сосредоточенных в рассматриваемом регионе. Такая модель позволяет при наличии данных о занятости по отраслям в регионе и в государстве быстро и эффективно строить балансы для отдельных субъектов федерации. Главным ограничением в данной модели является предположение о том, что структура производств одной отрасли в различных регионах одинакова. Однако, существует множество различных модификаций модели, позволяющих отходить от данного предположения.

Таким образом, хотя первый подход опирается на реальные статистические данные, имеющейся детализации недостаточно для того, чтобы полностью отказаться от привязки к федеральной модели, в то время как обработка существующей статистики достаточно затратна по времени и требует больших усилий при желании рассмотреть новый регион или выбрать другой год. Вторая модель, хотя и привязана к федеральной, и связана предположением о схожей структуре производственных процессов в различных регионах, позволяет быстро и достаточно точно построить матрицы межотраслевых балансов для различных регионов по небольшим объемам статистических данных.
\subsection{Матрица межотраслевого баланса для Новосибирской области}

Пока используются данные из \cite{MOB_loc_base}.
\section{Оценка влияния пандемии Covid-19 на занятость и трудоспособность населения в Новосибирской области}
\section{Влияние шоков от пандемии Covid-19 на экономику Новосибирской области}
\section{Заключение}
1. Построить симметричную матрицу для регионов
2. Построить модель межотраслевого баланса
3. Моделировать влияние шоков в экономике на ВРП
4. Оценить силу шоков по эпидемиологическим данным
